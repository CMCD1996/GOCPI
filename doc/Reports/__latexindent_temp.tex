% !TEX TS-program = pdflatex
% !TEX encoding = UTF-8 Unicode
% !BIB TS-program = biber
% !BIB program = biber

\documentclass[12pt]{article}

%%% PAGE DIMENSIONS
\usepackage[margin=2.54cm]{geometry}
\geometry{a4paper} 
% \usepackage{caption}
% \usepackage{subcaption}
\usepackage{graphicx} % For better graphics
\usepackage{pdfpages} % To insert pdfs into the library
\usepackage{tikz}
\usepackage{wrapfig}
\usepackage{siunitx}

%%% PACKAGES
\usepackage{booktabs} % for much better looking tables
\usepackage{amsmath} % for better maths
\usepackage{paralist} % very flexible & customisable lists (eg. enumerate/itemize, etc.)
\usepackage{verbatim} % adds environment for commenting out blocks of text & for better verbatim
\usepackage{subfig} % make it possible to include more than one captioned figure/table in a single float
\usepackage[framed,numbered]{matlab-prettifier} % enable inserting matlab code.
\usepackage[parfill]{parskip}
%\addtolength{\jot}{1em}
\usepackage{amssymb}
\usepackage{cancel}
\usepackage{color}
\usepackage{listings}
% Create Listing Colours
\usepackage{xcolor}

% \definecolor{codegreen}{RGB}{0,255,0}
% \definecolor{codegray}{RGB}{105,105,105}
% \definecolor{codepurple}{RGB}{138,43,226}
% \definecolor{backcolour}{RGB}{255,255,255}
\definecolor{codegreen}{rgb}{0,0.6,0}
\definecolor{codegray}{rgb}{0.5,0.5,0.5}
\definecolor{codepurple}{rgb}{0.58,0,0.82}
\definecolor{backcolour}{rgb}{0.95,0.95,0.92}

\lstdefinestyle{mystyle}{
    backgroundcolor=\color{backcolour},   
    commentstyle=\color{codegreen},
    keywordstyle=\color{magenta},
    numberstyle=\tiny\color{codegray},
    stringstyle=\color{codepurple},
    basicstyle=\ttfamily\footnotesize,
    breakatwhitespace=false,         
    breaklines=true,                 
    captionpos=b,                    
    keepspaces=true,                 
    numbers=left,                    
    numbersep=5pt,                  
    showspaces=false,                
    showstringspaces=false,
    showtabs=false,                  
    tabsize=2
}
\lstset{style=mystyle}

\usepackage{multicol}
\usepackage{float}

% References
\usepackage[backend=biber,style = numeric]{biblatex}
\bibliography{sources}

%%% HEADERS & FOOTERS
\usepackage{fancyhdr} % This should be set AFTER setting up the page geometry
\setlength{\headheight}{15pt}
\pagestyle{fancy} % options: empty , plain , fancy
\renewcommand{\headrulewidth}{0pt} % customise the layout...
\lhead{University of Auckland}\chead{GOCPI}\rhead{Connor McDowall}
\lfoot{}\cfoot{\thepage}\rfoot{}

%%% SECTION TITLE APPEARANCE
\usepackage{sectsty}

%%% ToC (table of contents) APPEARANCE
\usepackage[nottoc,notlof,notlot]{tocbibind} % Put the bibliography in the ToC
\usepackage[titles,subfigure]{tocloft} % Alter the style of the Table of Contents
\renewcommand{\cftsecfont}{\rmfamily\mdseries\upshape}
\renewcommand{\cftsecpagefont}{\rmfamily\mdseries\upshape} % No bold!

%%% Hyperlinking 
\usepackage{hyperref}
\begin{document}
\begin{titlepage}
	\newcommand{\HRule}{\rule{\linewidth}{0.5mm}} % Defines a new command for horizontal lines, change thickness here
	
	\center
	
	%------------------------------------------------
	%	Headings
	%------------------------------------------------
	
	\textsc{\LARGE }\\[1.5cm] % Main heading such as the name of your university/college
	
	\textsc{\Large ENGSCI 700A/B}\\[0.5cm] % Major heading such as course name
	
	%------------------------------------------------
	%	Title
	%------------------------------------------------
	
	\HRule\\[0.5cm]
	
	{\huge\bfseries Research Compendium README}\\[0.4cm] % Title of your document
	
	\HRule\\[0.5cm]
	
	%------------------------------------------------
	%	Author(s)
	%------------------------------------------------
	
	{\large\textit{Connor McDowall \\cmcd398 \\530913386}}\\
	
	%------------------------------------------------
	%	Date
	%------------------------------------------------
	
	\vfill\vfill\vfill % Position the date 3/4 down the remaining page
	
	{\large\today} % Date, change the \today to a set date if you want to be precise
	 
	%----------------------------------------------------------------------------------------
	
	\vfill % Push the date up 1/4 of the remaining page
	
\end{titlepage}
\section*{Declaration of Contribution}
I proposed this project. I am the sole contributor.

\tableofcontents
\listoffigures
\newpage
\section{Research Compendium Submission}
The zip file submitted includes the most relevant files for the project.
All project files are accessible on \textbf{\href{https://github.com/CMCD1996/GOCPI}{GitHub}}.
\subsection{Data}
This sub directory includes the data required for the project.
\begin{itemize}
    \item \textbf{IEAWorldEnergyBalances2017A-K.csv}: IEA energy balances for countries A-K. The file is too large data Energy Balances.
    \item \textbf{IEAWorldEnergyBalances2017L-Z.csv}: IEA energy balances for countries L-Z.
    \item \textbf{Geo\_EB.xlsx}: Energy balances (Reference energy system) for user defined region.
    \item \textbf{NZ MBIE Energy Balances.xlsx}: Energy statistics to design NZ reference energy system.
    \item \textbf{AUS GOV Energy Balances.xlsx}: Energy statistics to design AUS reference energy system.
\end{itemize}
\subsection{Reports}
This sub directory includes the documentation related to the project.
\begin{itemize}
    \item \textbf{ENGSCI 700AB Project Report.pdf}: Report containing Literature Review, Project Scope, Methodology, Implementation, Results, Summary and Conclusion.
    \item \textbf{ENGSCI 700AB Research Compendium.pdf}: Portable Document File (PDF) of most relevant project information.
    \item \textbf{GOCPI Documentation.pdf}: Documentation generated using docstrings and sphinx to describe the GOCPI package.
    \item \textbf{ENGSCI 700AB Project Logbook.pdf}: Logbook to track project progression.
\end{itemize}
\subsection{Source Code}
This sub directory contains all the source code related to the GOCPI project.
\subsubsection{GOCPI Package}
\begin{itemize}
	\item \textbf{CreateCases.py}: Module to create user-designed energy systems.
	\item \textbf{Energysystems.py}: Module to create the model and data files needed for solving energy systems.
	\item \textbf{Forecasting.py}: Module to forecast energy and finance-related data.
	\item \textbf{Navigation.py}: Module to navigate directories.
	\item \textbf{Optimisation.py}: Module to solve energy systems either locally or remotely using IBM Technologies.
\end{itemize}
\subsubsection{OseMOSYS Model}
Disclaimer: The LP file used formulated using the model and data file is too large and impractical to include.
It is formulated using the GNU Linear programming Kit. 
This file (GOCPI.lp) is found by going to data, Inputs then GOCPI OseMOSYS in the GitHub Repository.
\begin{itemize}
	\item \textbf{GOCPI\_OseMOSYS\_Data.txt}: The data file created from the processing script.
	\item \textbf{GOCPI\_OseMOSYS\_Model.txt}: The model file created from the GOCPI\_OseMOSYS\_Structure.xlsx script.
	\item \textbf{GOCPI\_OseMOSYS\_Structure.xlsx}: The spreadsheet storing the OseMOSYS Model.
\end{itemize}
\subsubsection{Processing Scripts}
These scripts helped design and build the GOCPI package and energy systems.
\begin{itemize}
	\item \textbf{GOCPI\_Data\_Cases.gyp}: Script to help design CreateCases module.
	\item \textbf{GOCPI\_EB.gyp}: Script to help formulate reference energy systems from IEA Data.
	\item \textbf{GOCPI\_Geographies.gyp}: Script to help create user defined regions.
	\item \textbf{GOCPI\_Inputs.gyp}: Script to help create a standardised TIMES modelling method.
	\item \textbf{GOCPI\_Model\_Import.gyp}: Script to import the OseMOSYS model from Excel.
	\item \textbf{GOCPI\_Optimisation.gyp}: Script to help solve energy systems using IBM Technologies.
	\item \textbf{GOCPI\_NZ\_Example.gyp}: Script to build NZ and AUS energy systems with a bi-lateral trade relationship.
\end{itemize}
\end{document}